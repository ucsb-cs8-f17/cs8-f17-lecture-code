\documentclass{beamer}

% for themes, etc.

 \usetheme{default} 

%\usepackage{times} % fonts are up to you
\usepackage{graphicx}
\usepackage{verbatim}
% these will be used later in the title page
\title{Processing data in files}



% have this if you'd like a recurring outline
\AtBeginSection[] % "Beamer, do the following at the start of every section"
{
\begin{frame}<beamer> 
\frametitle{Outline} % make a frame titled "Outline"
\tableofcontents[currentsection] % show TOC and highlight current section
\end{frame}
}

\begin{document}
\bibliographystyle{plain}
% this prints title, author etc. info from above
\begin{frame}
\titlepage
\end{frame}


\begin{frame}[fragile]
\frametitle{Text Files}
\begin{itemize}
\item So far, most of our programs have retrieved data from the keyboard and written data to the screen
\begin{itemize}
\item Data must be entered on every program run
\item Programs have no way to write permanent output
\end{itemize}
\item Text files provide convenient input/output storage
\item e.g.\ programs can read configuration data or input files to process, and can write output to files
\end{itemize}
\end{frame}

\begin{frame}[fragile]
\frametitle{ConcepTest}
A program is designed to retrieve some data from a file, process it, and output the revised data to another file. Which of the following functions/methods will {\bf not} be called in the program?

\begin{itemize}
\item A. \verb|open|
\item B. A loop or method for reading (e.g.\ \verb|read|)
\item C. \verb|write|
\item D. \verb|close|
\item E. All should be called
\end{itemize}
\end{frame}

\begin{frame}[fragile]
\frametitle{Reading Files with Methods}
Several methods for reading text from files:
\begin{itemize}
\item \verb|readline|: reads and returns next line; returns empty string at end-of-file
\item \verb|read|: reads the entire file into one string
\item \verb|readlines|: reads the entire file into a list of strings
\end{itemize}

All of these leave a trailing \verb|'\n'| character at the end of each line.
\end{frame}

\begin{frame}[fragile]
\frametitle{Reading Files with Loops}
A file is a sequence of lines:
\begin{small}
\begin{verbatim}
f = open('songs.txt')
for line in f:
  print(line.strip())
\end{verbatim}
\end{small}

$\ldots$ or using a while-loop:
\begin{small}
\begin{verbatim}
f = open('songs.txt')
line = f.readline()
while line:
  print(line.strip())
  line = f.readline()
\end{verbatim}
\end{small}
\end{frame}

\begin{frame}[fragile]
\frametitle{ConcepTest}
Here are the first few lines of \verb|songs.txt|:
\begin{small}
\begin{verbatim}
Songs Chosen this Semester
#Game names for the songs
#All ridiculously good soundtracks
Sep, 17, Chrono Cross
Sep, 19, Zelda 3
\end{verbatim}
\end{small}

\begin{scriptsize}
\verbatiminput{skip1.py}
\end{scriptsize}

\verb|f| refers to this file and has just been opened. What does \verb|skip| return?
\begin{itemize}
\item A. The first line
\item B. The second line
\item C. The third line
\item D. The fourth line
\item E. The fifth line
\end{itemize}
\end{frame}

\begin{frame}[fragile]
\frametitle{ConcepTest}
Here are the first few lines of \verb|songs.txt|:
\begin{small}
\begin{verbatim}
Songs Chosen this Semester
#Game names for the songs
#All ridiculously good soundtracks
Sep, 17, Chrono Cross
Sep, 19, Zelda 3
\end{verbatim}
\end{small}

\begin{scriptsize}
\verbatiminput{skip2.py}
\end{scriptsize}

\verb|f| refers to this file and has just been opened. What does \verb|skip| return?
\begin{itemize}
\item A. The first line
\item B. The second line
\item C. The third line
\item D. The fourth line
\item E. The fifth line
\end{itemize}
\end{frame}

\begin{frame}[fragile]
\frametitle{Multi-Field Records}
\begin{itemize}
\item So far, we have been reading entire lines from our file
\item But, our lines are actually records containing three fields: month name, day number, and game name
\item Let's write a function to read this data into three lists
\item The critical string method here is \verb|split|
\begin{itemize}
\item With no parameters, it splits around any space
\item With a string parameter, it splits around that string
 \end{itemize}
\end{itemize}
\end{frame}

\begin{frame}[fragile]
\frametitle{Writing to files}

\begin{itemize}
\item In order to write to a file, the file must be opened for writing
\verb|f = open('writeSongs.txt','w')|
\item \verb|write(s)|: writes the string to the file

\end{itemize}

\end{frame}

\begin{frame}[fragile]
\frametitle{Coding exercise: Replace words}
Create a new file that reads a given file and replaces all the occurrences of one word by another
\begin{small}
\begin{verbatim}
def replaceWord(infileName, outfileName, oldWord, newWord):
	'''
     create a new file with the name outfileName that has the
     same content as infileName, but every occurrence of
     oldWord replaced by newWord
	'''
\end{verbatim}
\end{small}


\end{frame}

\end{document}
